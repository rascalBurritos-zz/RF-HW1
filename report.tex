\documentclass{article}

%font settings
% \usepackage[utf8]{inputenc}
\usepackage{fontspec}
\usepackage{unicode-math}
\setmainfont{Merriweather}
\setmathfont[Scale=1.25]{Asana Math}

%figure settings
\usepackage{graphicx}
\graphicspath{{../Lab/images/},{../Prelab/images/}}
\usepackage[letterpaper, margin=1in]{geometry}
\usepackage{multicol}
\usepackage{float}


%mathcha includes
\usepackage{physics}
\usepackage{amsmath}
\usepackage{tikz}
\usepackage{mathdots}
% \usepackage{yhmath}
\usepackage{cancel}
\usepackage{color}
\usepackage{siunitx}
\usepackage{array}
\usepackage{multirow}
% \usepackage{amssymb}
\usepackage{gensymb}
\usepackage{tabularx}
\usepackage{booktabs}
\usetikzlibrary{fadings}
\usetikzlibrary{patterns}
\usetikzlibrary{shadows.blur}
\usetikzlibrary{shapes}


\begin{document}
\begin{titlepage}
  \begin{center}
  \vspace*{1in}    
  \Huge
  \textbf{EE3302, Section 306}

  \textbf{Laboratory Fundamentals II}
  \vspace*{0.25in}    

  \Large
  Lab Topic: BJT AC Analysis

  Name: Stephen Campbell

  Instructor: Kamran Kiasaleh

  Date : \today

  \end{center}
\end{titlepage}
\begin{abstract}
 In this lab, the AC characteristics and small signal model of the BJT is analyzed,
 quantified, and measured. First, a common emitter amplifier is designed, constructed
 and analyzed. The frequency response is measured and its transient characteristics 
 are recorded. Afterwards, A differential long-tailed pair amplifier was designed
 and constructed. The characteristics of this system were measured and compared to 
 the theoretical values.
\end{abstract}
\section*{Introduction}
The design of the common emitter amplifier was performed under the guidance of
the lab manual and introduction. By using rules of thumbs and performing simulations,
the circuit with the desired gain and frequency response was specified and constructed.
The differential amplifier was made with a similar guise. Using the BJT equivalent model 
with equations listed in the manual, a differential long tailed pair amplifier was
exhibited was the desired characteristics. The important characteristics of 
the differential amplifier includes the differential gain as well as the input and 
output impedances.
\section*{Procedure}
The procedure consisted of constructing 2 circuits: a common emitter amplifier, a
differential amplifier
\begin{enumerate}
  \item \textbf{Common Emitter Amplifier}

  The common emitter amplifier designed and simulated in the prelab was constructed with on hand 
  components. With the amplifier constructed, the network analyzer features of the 
  ADALM2000 was utilized to characterize the frequency response of the amplifier. 
  The measured frequency response allowed one to calculate the gain of the amplifier.
  This measured gain was compared to the theoretical expected value of the gain which
  was determined by the simulations and circuit analysis. 
  \item \textbf{Differential Amplifier}

  The differential amplifier designed and simulated in the prelab was constructed with on hand 
  components. The transient response of the amplifier was measured by applying 2 
  sinusoids that were out of phase by 180 degrees to the amp inputs and by measured the
  voltage across the collectors. The input impedance was measured by applying a 
  differential voltage across the inputs of the amplifier and measured the current 
  through the inputs.
\end{enumerate}

\section*{Prelab Simulations and Results}
\subsection*{Problem 1}
\textbf{
Repeat example 1, but assume an amplifier with a gain of -8 which has identical 3
dB corner frequencies of 10 kHz for high pass coupling at the input and output.
Use a power supply of 5 volts.}


\begin{align*}
G & =-8\\
f_{i} & =f_{o} =10^{4}\text{Hz}\\
V_{CC} & =5\text{V}
\end{align*}
\begin{align*}
 & \text{Two Arbitrary Initial Conditions}\\
R_{C} & =800\Omega \\
R_{2} & =1000\Omega \\
 & \\
 & \text{Now,}\\
R_{E} & =\frac{R_{C}}{| G| }\\
R_{E} & =100\\
 & \text{Using a load line 
to determine R1}\\
I_{C} & =-\frac{1}{R_{E} +R_{C}}( V_{CE} -V_{CC})\\
y_{\text{int}} & =\frac{V_{CC}}{R_{E} +R_{C}}\\
y_{\text{int}} & =\frac{5}{900} =5.55\ 10^{-3}\\
 & \text{Arbitrary Decision
on load line}
\end{align*}




\begin{align*}
V_{CE} & =1.5\text{V}\\
I_{B} & =20\times 10^{-6}\\
 & \text{Now,}\\
I_{C} & =3\times 10^{-3}\text{A}\\
\beta  & =\frac{3\times 10^{-3}}{20\times 10^{-6}}\\
\beta  & =150\\
\beta  & \gg 1\\
I_{E} & =I_{C} +I_{B}\\
I_{E} & =3\times 10^{-3} +20\times 10^{-6}\\
I_{E} & =3.02\times 10^{-3}\\
V_{B} & =0.7+R_{E} I_{E}\\
V_{B} & =0.7+( 100) 3.02\times 10^{-3}\\
V_{B} & =1.002\\
\frac{V_{CC} -V_{B}}{R_{1}} & =I_{B} +\frac{V_{B}}{R_{2}}\\
\frac{5-1.002}{R_{1}} & =20\times 10^{-6} +\frac{1.002}{1000}\\
\frac{3.998}{R_{1}} & =1.022\ \times 10^{-3}\\
R_{1} & =3911.94\Omega 
\end{align*}
\begin{align*}
 & \text{To determine cutoff frequency for 
input coupling cap}\\
R_{ei} & =R_{1} \parallel R_{2} \parallel \beta R_{e}\\
f_{i} & =\frac{1}{2\pi R_{ei} C_{i}}\\
C_{i} & =\frac{1}{2\pi R_{ei} f_{i}}\\
R_{ei} & =\left(\frac{1}{3900} +\frac{1}{1000} +\frac{1}{15000}\right)^{-1}\\
R_{ei} & =755.814\\
C_{i} & =\frac{1}{2\pi ( 755.814)\left( 10^{4}\right)}\\
C_{i} & =2.1\ \times 10^{-8}\text{F}\\
 & \text{Now the output cap}\\
f_{o} & =\frac{1}{2\pi ( R_{C} +R_{L}) C_{0}}\\
C_{0} & =\frac{1}{2\pi ( R_{C} +R_{L}) f_{0}}\\
C_{0} & =\frac{1}{2\pi ( 850) 10^{4}}\\
C_{0} & =1.87241\times 10^{-8}\text{F}
\end{align*}
In summary,


\begin{align*}
R_{1} & =3911.94\\
R_{2} & =1000\Omega \\
R_{C} & =800\Omega \\
R_{E} & =100\Omega \\
C_{i} & =2.1\ \times 10^{-8}\text{F}\\
C_{o} & =1.87241\times 10^{-8}\text{F}
\end{align*}


\subsection*{Problem 2}
\textbf{
Use PSpice to model the amplifier in part 1. Do an AC sweep of the circuit for a
frequency range 1 kHz-40 kHz. Show the gain of the circuit over this frequency
band. Comment on your observation (make sure that you use a small enough input
signal that does not result in a nonlinear behavior for the amplifier). 
}
\begin{figure}[H]
  \centering
  \includegraphics[height=3in]{BJTAMP-Schematic-NoLoad.png}
  \caption{BJT Amplifier Circuit Schematic}
\end{figure}
\begin{figure}[H]
  \centering
  \includegraphics[height=3in]{BJTAMP-ACsweep-NoLoad.png}
  \caption{Input and Output Voltages of AC sweep (1kHz to 40kHz) of BJT amp with no load }
\end{figure}
\begin{figure}[H]
  \centering
  \includegraphics[height=3in]{BJTAMP-ACsweep-NoLoadGain.png}
  \caption{Gain of BJT Amp  no load : AC sweep (1kHz to 40kHz) }
\end{figure}
\begin{figure}[H]
  \centering
  \includegraphics[height=3in]{BJTAMP-ACsweep-Load.png}
  \caption{Input and Output Voltages of AC sweep (1kHz to 40kHz) of BJT amp with load }
\end{figure}
\begin{figure}[H]
  \centering
  \includegraphics[height=3in]{BJTAMP-ACsweep-LoadGain.png}
  \caption{Gain of BJT Amp with load : AC sweep (1kHz to 40kHz) }
\end{figure}

% Without a load, the gain of the Amp seems to increase to the desired
% gain through increasing the frequency. 
At around 40kHz the gain seems to be around -8.  Not evident on these graphs, is the 180 degree phase
difference. This would make the +8 apparent gain to a -8 gain.With the load, the gain 
seems to max out at around -6dB. The high pass effects are as desired.

\subsection*{Equivalent BJT}
\textbf{
Find an equivalent model for your BJT by examining the datasheet for 2N3904.
You should be able to fund Cu and Cpi from the datasheet. They are named
differently in the datasheet. Use the operating point (iC) obtained in steps 1 and 2.
For VA use the IV characteristics of the BJT which you had found in the previous
experiment
}

On the ST datasheet for the 2N3904 the collector base capacitance
 is 4pF and the emitter base capacitance is 18pF. The collector 
 base capacitance is the same as $C_{\mu}$ and the emitter base 
 capacitance is the same as $C_{\pi}$.
% On the datasheet, Input Capacitance = 8pF, Output Capacitance 4pF

\begin{align*}
C_{\pi } & =18\times 10^{-12}\text{F}\\
C_{\mu } & =4\times 10^{-12}\text{F}\\
g_{m} & =\frac{I_{C}}{V_{T}}\\
g_{m} & =\frac{3\times 10^{-3}}{0.026}\\
g_{m} & =0.115385\\
r_{e} & =g^{-1}_{m}\\
r_{e} & =8.66664\\
r_{\pi } & =\frac{\beta }{g_{m}}\\
r_{\pi } & =\frac{150}{0.115385}\\
r_{\pi } & =1300
\end{align*}

\begin{align*}
 & \text{Points from the Measured } I_C \text{ vs. } V_{CE}\\
( V_{CE} ,I_{C}) & =( 0.788,0.009759)\\
( V_{CE} ,I_{C}) & =( 3.403,0.009997)\\
m & =\frac{0.009997-0.009759}{3.403-0.788}\\
m & =9.101\times 10^{-5}\\
y & =m( x-V_{a}) +y_{0}\\
y & =mx-mV_{a} +y_{0}\\
y-mx & =-mV_{a} +y_{0}\\
( 0.009759) -\left( 9.101\times 10^{-5}\right)( 0.788) & =-9.101\times 10^{-5} V_{a} +y_{0}\\
( 0.009997) -\left( 9.101\times 10^{-5}\right)( 3.403) & =-9.101\times 10^{-5} V_{a} +y_{0}\\
9.68728\times 10^{-3} & =-9.101\times 10^{-5} V_{a} +y_{0}\\
9.68729\times 10^{-3} & =-9.101\times 10^{-5} V_{a} +y_{0}\\
V_{a} & =106.438\\
y_{0} & =9.68728\times 10^{-3}\\
r_{0} & =\frac{V_{A}}{I_{C}}\\
r_{0} & =\frac{106.438}{3\times 10^{-3}}\\
r_{0} & =35.479\text{k} \Omega 
\end{align*}

\begin{figure}[H]
  \centering
  \includegraphics[height=3in]{Eq-BJT-Schematic.png}
  \caption{Equivalent BJT circuit schematic}
\end{figure}

\subsection*{DIBO Differential Amp}

\textbf{
Use PSpice to model the differential amplifier circuit shown in Fig. 4 in DIBO mode.
Use 2N3904 BJTs and use appropriate values for resistors (you can choose the
values that will not lead to excessive gain and saturation) to demonstrate that the
circuit provides differential amplification. Use Vcc = 5 and Vee = 5. Use a pair of
sinusoids with opposing polarity (180 degree phase shift) as the inputs to the
differential amplifier. Recall from the theory Ic 
is needed to compute re
. Make sure that the conditions set in the analysis of DIBO circuit are satisfied. 
Assume RS1 = RS2 = 50 Ohm.
}

\begin{align*}
 & \text{Conditions}\\
R_{1} & =R_{2} =R_{L}\\
r_{e1} & =r_{e2} =r_{e}\\
\frac{R_{s1}}{\beta } ,\frac{R_{s2}}{\beta } & \ll R_{3} ,r_{e}\\
 & \text{Design}\\
i_{C} & =\frac{V_{ee} -V_{BE}}{2R_{3}}\\
R_{3} & =\frac{V_{ee} -V_{BE}}{2i_{C}}\\
i_{C} & \rightarrow 2\times 10^{-3}\\
R_{3} & \rightarrow \frac{5-0.7}{2\left( 2\times 10^{-3}\right)} =1075\\
g_{m} & =\frac{I_{C}}{V_{T}} =\frac{2\times 10^{-3}}{0.026}\\
g_{m} & =7.692\times 10^{-2}\\
r_{e} & =\frac{1}{g_{m}} =\frac{1}{7.692\times 10^{-2}} =13\\
\text{Gain} & =4\\
R_{L} & =R_{1} =R_{2} =13\cdot 4=52
\end{align*}

\section*{Experimental Results}
\subsection*{Common Emitter Amplifier}
\begin{figure}[H]
  \centering 
  \includegraphics[height=3in]{BJTAMP-Schematic.png} 
  \caption{ BJT Common Emitter Amplifier Schematic}
\end{figure}
\begin{figure}[H]
  \centering 
  \includegraphics[width=\textwidth]{BJTAMP-Frequency-Response-Sim-NoLoad.png} 
  \caption{Simulation of BJT Common Emitter Amplifier Frequency Response}
\end{figure}
\begin{figure}[H]
  \centering 
  \includegraphics[height=3in]{BJTAMP-FrequencyResponseNoLoad.png} 
  \caption{Measured BJT Common Emitter Amplifier Frequency Response}
\end{figure}

\begin{figure}[H]
  \centering 
  \includegraphics[height=3in]{BJTAMP-FrequencyResponseLoad.png} 
  \caption{Measured BJT Common Emitter Amplifier Frequency Response with 50$\Omega$ Load}
\end{figure}


\textbf{
 Does your experimental results match your simulation results in the
pre-lab? If not, explain the possible causes for the discrepancy.
}

\vspace{1em}
The simulation results match the measured values of the frequency
response without load. With a load the expected values is also seen.
In the loaded case, the gain of the amplifier is the output impedance over
the equivalent emitter resistance. In this case that is approximately $50/100$,
the the expected gain is 0.5 or -6.02dB. The measurements are roughly on the same 
order of this predicted gain. Some discrepancies in the measurement could be due to:
\begin{itemize}
  \item Resistor tolerances in both the load and emitter legs
  \item Large deviation from the operating point
  \item The capacitive effects in both the equivalent model and in the coupling
  capacitors
\end{itemize}





\subsection*{Differential Amplifier}
\begin{figure}[H]
  \centering 
  \includegraphics[height=3in]{DIFFAMP-MeasuredTransients.png} 
  \caption{Transient Differential Amplifier Measurements}
\end{figure}
% \begin{figure}[H]
%   \centering 
%   \includegraphics[height=3in]{} 
%   \caption{Input Current and Input Voltage of Differential Amplifier}
% \end{figure}

\textbf{
Does your measurement of gain and input impedance match the
theoretical and simulation results? If not, provide possible
reasons for the discrepancy.
}

\vspace{1em}
Through the transient measurement, the gain was recorded as $\approx 4$ which coincides
phenomenally with the theoretical gain of the amplifier as well as the simulation
results where the gain was 4. The theoretical input impedance of the differential amplifier 
is  $R_{\text{in}} = 2\beta r_e$. This corresponds to $\approx 2(225)(13) = 5850$. The
measured value of the input impedance was calculated with a 100mV input voltage and a
18.6$\mu$A current. This yields a measured input impedance of $\approx 5347.58 \Omega$.
The measured values closely matches this simulated and theoretical results. However,
the measurement of this value was highly sensitive to the resistor values, the operating 
point, the similarity of the BJTs, and many other factors. 

\end{document}