\documentclass{article}

%font settings
% \usepackage[utf8]{inputenc}
\usepackage{fontspec}
\usepackage{unicode-math}
\setmainfont{Merriweather}
\setmathfont[Scale=1.25]{Asana Math}

%figure settings
\usepackage{graphicx}
\graphicspath{{../Lab/images/}}
\usepackage[letterpaper, margin=1in]{geometry}
\usepackage{multicol}
\usepackage{float}


%mathcha includes
\usepackage{physics}
\usepackage{amsmath}
\usepackage{tikz}
\usepackage{mathdots}
% \usepackage{yhmath}
\usepackage{cancel}
\usepackage{color}
\usepackage{siunitx}
\usepackage{array}
\usepackage{multirow}
% \usepackage{amssymb}
\usepackage{gensymb}
\usepackage{tabularx}
\usepackage{booktabs}
\usetikzlibrary{fadings}
\usetikzlibrary{patterns}
\usetikzlibrary{shadows.blur}
\usetikzlibrary{shapes}


\begin{document}
\begin{titlepage}
  \begin{center}
  \vspace*{1in}    
  \Huge
  \textbf{EE3302, Section 306}

  \textbf{Laboratory Fundamentals II}
  \vspace*{0.25in}    

  \Large
  Lab Topic: BJT AC Analysis

  Name: Stephen Campbell

  Instructor: Kamran Kiasaleh

  Date : \today

  \end{center}
\end{titlepage}
\begin{abstract}
 In this lab, the AC characteristics and small signal model of the BJT is analyzed,
 quantified, and measured. First, a common emitter amplifier is designed, constructed
 and analyzed. The frequency response is measured and its transient characteristics 
 are recorded. Afterwards, A differential long-tailed pair amplifier was designed
 and constructed. The characteristics of this system were measured and compared to 
 the theoretical values.
\end{abstract}
\section*{Introduction}
The design of the common emitter amplifier was performed under the guidance of
the lab manual and introduction. By using rules of thumbs and performing simulations,
the circuit with the desired gain and frequency response was specified and constructed.
The differential amplifier was made with a similar guise. Using the BJT equivalent model 
with equations listed in the manual, a differential long tailed pair amplifier was
exhibited was the desired characteristics. The important characteristics of 
the differential amplifier includes the differential gain as well as the input and 
output impedances.
\section*{Procedure}
The procedure consisted of constructing 2 circuits: a common emitter amplifier, a
differential amplifier
\begin{enumerate}
  \item \textbf{Common Emitter Amplifier}

  The common emitter amplifier designed and simulated in the prelab was constructed with on hand 
  components. With the amplifier constructed, the network analyzer features of the 
  ADALM2000 was utilized to characterize the frequency response of the amplifier. 
  The measured frequency response allowed one to calculate the gain of the amplifier.
  This measured gain was compared to the theoretical expected value of the gain which
  was determined by the simulations and circuit analysis. 
  \item \textbf{Differential Amplifier}

  The differential amplifier designed and simulated in the prelab was constructed with on hand 
  components. The transient response of the amplifier was measured by applying 2 
  sinusoids that were out of phase by 180 degrees to the amp inputs and by measured the
  voltage across the collectors. The input impedance was measured by applying a 
  differential voltage across the inputs of the amplifier and measured the current 
  through the inputs.
\end{enumerate}

\section*{Prelab Simulations and Results}

\section*{Experimental Results}
\subsection*{Common Emitter Amplifier}
\begin{figure}[H]
  \centering 
  \includegraphics[height=3in]{BJTAMP-Schematic.png} 
  \caption{ BJT Common Emitter Amplifier Schematic}
\end{figure}
\begin{figure}[H]
  \centering 
  \includegraphics[width=\textwidth]{BJTAMP-Frequency-Response-Sim-NoLoad.png} 
  \caption{Simulation of BJT Common Emitter Amplifier Frequency Response}
\end{figure}
\begin{figure}[H]
  \centering 
  \includegraphics[height=3in]{BJTAMP-FrequencyResponseNoLoad.png} 
  \caption{Measured BJT Common Emitter Amplifier Frequency Response}
\end{figure}
\textbf{
 Does your experimental results match your simulation results in the
pre-lab? If not, explain the possible causes for the discrepancy.
}

\vspace{1em}
The simulation results match the measured values of the frequency
response.


\subsection*{Differential Amplifier}
\begin{figure}[H]
  \centering 
  \includegraphics[height=3in]{DIFFAMP-MeasuredTransients.png} 
  \caption{Transient Differential Amplifier Measurements}
\end{figure}
% \begin{figure}[H]
%   \centering 
%   \includegraphics[height=3in]{} 
%   \caption{Input Current and Input Voltage of Differential Amplifier}
% \end{figure}

\textbf{
Does your measurement of gain and input impedance match the
theoretical and simulation results? If not, provide possible
reasons for the discrepancy.
}
Through the transient measurement, the gain was recorded as $\approx 4$ which coincides
phenomenally with the theoretical gain of the amplifier as well as the simulation
results where the gain was 4. 
\end{document}