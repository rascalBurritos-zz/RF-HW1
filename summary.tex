\documentclass{article}


%misc
\usepackage{hyperref}

%font settings
\usepackage{fontspec}
\usepackage{unicode-math}
\setmainfont{Merriweather}
\setmathfont[Scale=1.25]{Asana Math}

%figure settings
\usepackage{graphicx}
\graphicspath{./images/}
\usepackage[letterpaper, margin=1in]{geometry}
\usepackage{multicol}
\usepackage{float}


%mathcha includes
\usepackage{physics}
\usepackage{amsmath}
\usepackage{tikz}
\usepackage{mathdots}
\usepackage{cancel}
\usepackage{color}
\usepackage{siunitx}
\usepackage{array}
\usepackage{multirow}
\usepackage{gensymb}
\usepackage{tabularx}
\usepackage{booktabs}
\usetikzlibrary{fadings}
\usetikzlibrary{patterns}
\usetikzlibrary{shadows.blur}
\usetikzlibrary{shapes}


\begin{document}

\begingroup  
    \centering
    \LARGE HW Set \#1\\[0.5em]
    \large \today\\[0.5em]
    \large Stephen Campbell\par
    \large Dr. Lehmann \par
    \large EE 4368 RF Design Principles\par
\endgroup
\rule{\textwidth}{0.4pt}


\section*{5G Testbed at the University of Tennessee, Knoxville}


%Definitions
%source https://spectrum.ieee.org/5g-bytes-millimeter-waves-explained#toggle-gdpr
\subsection*{Definitions}
\begin{itemize}
  \item \textbf{Millimeter wave spectrum:}

  Refers to a band of frequencies that wireless
  technology manufactures are experimenting with in order to overcome the saturation
  that exists on historically used wireless frequency bands. These waves are 
  typically broadcast between 30 and 300 GHz
  \item \textbf{Multi-Access Edge Computing}

  The practice of placing computing infrastructure physically closer to the end user to 
  gain ultra-low latency. "Edge" refers to the fact that these systems are not necessary
  to core infrastructure i.e. systems built with edge infrastructure are simply there to
  boost a system's responsiveness.
\end{itemize}

% source https://www.ibm.com/cloud/blog/what-is-multi-access-edge-computing


\subsection*{Summary}

% article source 
%paragraph
AT\&T is collaborating with the University of Tennessee, Knoxville (U.T.) to bring millimeter wave spectrum
(5G+) cell technology and multi-access edge computing to the university campus. AT\&T claims this will have major 
implications for the region and beyond : inspire agricultural innovation, enhance academic experiences, and provide ``warfighters''
the ability to ``see through'' walls. Farmer's can leverage this increased connectivity to boost 
their real time insights into their ``soil and crop health".
Students at UT  can participate in 5G+ enabled learning experiences with both
augmented reality and virtually reality.These digital experiences 
can emulate dangerous or impractical scenarios that could be potential useful
for education e.g. touring a nuclear power plant.  
Additionally, AT\&T noted that the 5G+ infrastructure
will enable machine learning algorithms to sift through students' class engagement and biometric data to evaluate performance and optimize the learning experience for the student.
The military hopes to utilize the millimeter spectrum with radar technology. This application would allows the militants to take pictures through walls and communicate
that information to allies. Overall,
AT\&T and a cross functional team of professors at U.T.
will research, experiment, and test a real world implementation of
5G+ and multi-access edge computing technology to gain insight into how to improve this cutting edge engineering.

% \subsection*{Personal Thoughts}
% Overall, I feel this net benefit to society; Enabling enhanced connectivity can be incredibly beneificial for a rural area.
% Some potential impact areas that were listed did raise some red flags.
% Utilizing this technology to analyze student biometric data immidietly triggers privacy concerns. The intention, personalizing learning experience,
% seems not immeditely related. 

\subsection*{Source}

\begin{itemize}
  \item Title: The University of Tennessee and AT\&T are Bringing 5G to the
University’s Knoxville Campus to Power Research, Education and
Innovation
  \item Name of News Media: AT\&T Corporate Communications
  \item Author: Andrea Huguely
  \item Date: August 17, 2021
  \item Website Link: \url{https://about.att.com/story/2021/the_university_of_tennessee_5g.html}
\end{itemize}

\section*{NFC Textiles}

%definitions

\subsection*{Definitions}

%source https://en.wikipedia.org/wiki/Near-field_communication
\begin{itemize}
  \item \textbf{Near Field Communication (NFC):}

  Inductive coupling based technology which allows for contactless exchange of data <2cm. 
  It utilizes the frequency 13.56 MHz to interface at data rates between 106-424 kbit/s.
\end{itemize}


%article source : https://www.rfidjournal.com/researchers-develop-smart-furniture-with-flexible-nfc

\subsection*{Summary}
Carnegie Mellon's WiTech Lab are experimenting with variations of a Near Field Communication (NFC) reading system.
Traditionally, NFC systems have a wireless range of <2cm; WiTech's system utilizes a ``beamforming" algorithm
which allows their detection system to have a range of 20cm.
Benefitting from this range, their system produces 3D "localization" of NFC tags and other conductive objects. 
This localization elaborates the 3D position 
of these objects in space.
The WiTech lab uses a ``blind beamforming" algorithm to detect conductive objects that are 
not NFT tags. This technique complements the more sophisticated algorithms that detect and identify
actual NFT tags and allow for users to be uniquely identified or personalized smart home actions to be triggered.
WiTech's initial application of this experimentation is a couch equipped with textile 
antennae used together as a long range NFC detector. 
With this information in this application, their system can identify someone's posture e.g lying down 
or sitting on a couch to trigger different actions.
 WiTech's team iterates that
further optimizations can be performed to enhance the system. Namely, 
increasing the 20cm range and implementing a 
form of encryption to bolster the system's security.

\subsection*{Source}
\begin{itemize}
  \item Title: Researchers Develop Smart Furniture with Flexible NFC
  \item Name of News Media: RFID Journal
  \item Author: Claire Swidberg
  \item Date: August 18, 2021
  \item Website Link: \url{https://www.rfidjournal.com/researchers-develop-smart-furniture-with-flexible-nfc}
\end{itemize}

\section*{Shielding Ultracold Molecules with Microwaves}

\subsection*{Summary}
Quantum technologies like quantum computing and quantum simulation 
utilize so called ``ultracold molecules".
A key drawback of utilizing these molecules is that molecule-molecule
collision essentially eliminates these molecules' effectiveness.
Reducing these collisions is one of the paramount challenges that 
quantum technology needs to overcome in order to be useful.
An international team of researchers has shown that these collisions
can be reduced by ``guiding the interaction between molecules using
microwaves". The dipole moments of these molecules have 2 major practical interactions.
Namely, their oscillations can be synchronized and controlled to lab created microwaves and 
the dipole moments between adjacent molecules can interact with one another
causing repulsion or attraction. With these interactions in mind, the researchers engineered a method
to repel individual calcium monofluoride molecules at 100 $\mu$K. This reduced the collisions 
and associated loss rate six fold. Furthermore, this microwave ``shielding" 
has been show to boost elastic collisions 17-fold. This shielding and elastic collisions 
are beneficial for further cooling making the goal of a ultracold molecule
gas even closer to realization.

\subsection*{Source}
\begin{itemize}
  \item Title: Shielding ultracold molecules with microwaves
  \item Name of News Media: Phys.org
  \item Author: Radboud University
  \item Date: August 13, 2021 
  \item Website Link: \url{https://phys.org/news/2021-08-shielding-ultracold-molecules-microwaves.html}
\end{itemize}

\end{document}
