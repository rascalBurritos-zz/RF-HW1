\documentclass{article}

%font settings
\usepackage{fontspec}
\usepackage{unicode-math}
\setmainfont{Merriweather}
\setmathfont[Scale=1.25]{Asana Math}

%figure settings
\usepackage{graphicx}
\graphicspath{./images/}
\usepackage[letterpaper, margin=1in]{geometry}
\usepackage{multicol}
\usepackage{float}


%mathcha includes
\usepackage{physics}
\usepackage{amsmath}
\usepackage{tikz}
\usepackage{mathdots}
\usepackage{cancel}
\usepackage{color}
\usepackage{siunitx}
\usepackage{array}
\usepackage{multirow}
\usepackage{gensymb}
\usepackage{tabularx}
\usepackage{booktabs}
\usetikzlibrary{fadings}
\usetikzlibrary{patterns}
\usetikzlibrary{shadows.blur}
\usetikzlibrary{shapes}


\begin{document}
\begin{titlepage}
  \begin{center}
  \vspace*{1in}    
  \Huge
  \textbf{EE4368, Section 001}

  \textbf{RF Circuit Design Principles}
  \vspace*{0.25in}    

  \Large
  HW Set #1 %LAB TITLE%!!!!!!!!!!!!!!!!!

  Name: Stephen Campbell

  Instructor: Randall Lehmann

  Date : \today

  \end{center}
\end{titlepage}
% \begin{abstract}
% \end{abstract}
\section*{5G Testbed at the University of Tennessee, Knoxville}


%Defitinions
%source https://spectrum.ieee.org/5g-bytes-millimeter-waves-explained#toggle-gdpr
millimeter wave spectrum: Refers to a new band of frequencies that wireless
technology manufactures are expirementing with in order to overcome the satuartion
that exists on historiccally used wirless frequency bands. These waves are 
typically broadcast between 30 and 300 GHz


Multi-Access Edge Computing
% source https://www.ibm.com/cloud/blog/what-is-multi-access-edge-computing
The practice of placing computing infrastructure physically closer to the end user to 
gain ultra-low latency. "Edge" refers to the fact that these systems are not necessary
to core infrastructure i.e. systems built with edge infrastructure are simply there to
boost a system's responsivness.



% article source
https://about.att.com/story/2021/the_university_of_tennessee_5g.html
%paragraph
AT\&T is colloborating with the University of Tennesse, Knoxville (U.T.) to bring millimeter wave spectrum
(5G+) cell technology and Multi-Access Edge Computing to the University Campus. AT\&T claims this will have major 
implications for the region and beyond : inspire agricultural innovation, enhance academic experiences, and provide "warfighters"
the ability to "see through" walls. Farmer's can leverage this  increased connectivity to boost 
their real time insights into their "soil and crop health".
Students at UT  can participate in 5G+ enabled learning experiences with both
augmented reality and virtualy reality. Additionally, the 5G+ infrastructure
will enable machine learning algorithms to sift through student engagement and biometric data to evaluate performance and optimize the learning experience for the student.
The military hopes to utilize t he millimeter spectrum with radar technology. This application would allows the militants to take pictures through walls and communicate
that information to allies. Overall, AT/&T and a cross functional team of professors at U.T. will  research, expiriment, and test a real world implementation of 5G  Multi-Access Compute Technology.

\subsection*{Personal Thoughts}
Overall, I feel this net benefit to society; Enabling enhanced connectivity can be incredibly beneificial for a rural area.
Some potential impact areas that were listed did raise some red flags.
Utilizing this technology to analyze student biometric data immidietly triggers privacy concerns. The intention, personalizing learning experience,
seems not immeditely related. 

\section*{NFC Textiles}

%definitions

Near Field Communication (NFC)
%source https://en.wikipedia.org/wiki/Near-field_communication
Inductive coupling based technology which allows for contactless exchange of data over <2cm. 
It utilizes the frequency 13.56 MHz to interface at data rates between 106-424 kbit/s.


%article source : https://www.rfidjournal.com/researchers-develop-smart-furniture-with-flexible-nfc
Carnegie Mellon's WiTech Lab are expirementing with a Near Field Communication reading system.
This system allows for the detection and positioning of NFC tags and other conductive object in 3D space.
Traditionally, NFC systems have a wireless range of <2cm; WiTech's system utilizes a "beamforming" algorithm
which allows their detection system to have a range of 20cm. WiTech is applying this 
technology to the home in the form of furniture. Their initial application is to have a couch equipped with textile 
antennae and use it as the detector. 
The couch could use a "blind beamforming" algorithm to detect conductive object that are 
not NFT tags. More sophisiticated alogorithms could be utlized with NFT tags that could
allow for users to be identified or smart home actions to be triggered. 


 




\section*{Discussion and Conclusion}
\end{document}
